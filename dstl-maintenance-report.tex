\documentclass{llncs}

\usepackage{graphicx}
\usepackage{url}
\usepackage{courier}
\usepackage{listings}
\usepackage{enumerate}


%%!TEX root = ./paper1.tex

\lstset{
  float=tb,
	captionpos=b,
	breaklines=true,
	xleftmargin=20pt,
	basicstyle=\ttfamily\scriptsize,
	numberstyle=\tiny,
	flexiblecolumns=true,
	numbers=left,
	nolol=false,
	tabsize=2
}

\lstdefinelanguage{OCL}{
morekeywords={import,if,then,else,endif,self,and,true,false,def,includes,OclElement,package,let,in},
sensitive=true,
morecomment=[l]{--},
morestring=[b]",
morestring=[b]',
showstringspaces=false
}

\lstdefinelanguage{QVTo}{
morekeywords={import,modeltype,uses,transformation,inout,in,out,configuration,property,main,var,if,then,else,endif,map,new,self,library,helper, mapping, and,return, when, where, object, true, false, result},
sensitive=true,
morecomment=[l]{--},
morecomment=[l]{//},
morestring=[b]",
morestring=[b]',
showstringspaces=false
}

\lstdefinelanguage{Acceleo}{
morekeywords={template, file, if, else, for},
sensitive=true,
morecomment=[l]{--},
morecomment=[l]{//},
morestring=[b]",
morestring=[b]',
showstringspaces=false
}

\lstdefinelanguage{MWE}{
morekeywords={module, import, var, true, false, },
sensitive=true,
morecomment=[l]{//},
morestring=[b]",
showstringspaces=false
}

\lstdefinelanguage{Java}{
morekeywords={class, private, public, true, false, new, if, for, int, return, void, extends, implements, this, null, super, import, package},
sensitive=true,
morecomment=[l]{//},
morestring=[b]",
showstringspaces=false
}

\lstdefinelanguage{JastAdd}{
morekeywords={abstract, ast, syn, inh, eq, boolean, int, false, true, if, for, return},
morestring=[b]',
sensitive=true
}

\lstdefinelanguage{NaBL}{
morekeywords={rules, defines, unique, non, refers, to},
sensitive=true
}

\lstdefinelanguage{Gra2Mol}{
morekeywords={rule, from, to, queries, mappings, skip, end_rule},
morestring=[b]',
sensitive=true
}

\lstdefinelanguage{Xtext}{
morekeywords={terminal, returns, grammar, import, fragment, current},
morestring=[b]',
morecomment=[l]{//},
sensitive=true
}

\lstdefinelanguage{Xtend}{
morekeywords={FOR, ENDFOR, IF, ELSE, ENDIF, def, protected, void, new, var, typeof, return},
morestring=[b]',
morestring=[b]",
morecomment=[s]{/*}{*/},
sensitive=true
}

\lstdefinelanguage{CS2AS}{
morekeywords={source, target, nameresolution, named, element, exports, for, from, all, children, resolution, helpers, mappings, map, disambiguation, lookup, lookupFrom, resolve, trace, when, occluding, nested,scope, def, protected, import,if,then,else,endif,self,and,true,false,def,includes,OclElement,package,let,in, void, new, var, typeof, return},
morestring=[b]',
morestring=[b]",
morecomment=[s]{/*}{*/},
sensitive=true
}



\begin{document}


\title{Technical Obsolescence Management Strategies for Safety-Related Software for Airborne Systems -- Long-Term Support}

\author{Richard F. Paige\inst{1}, Simos Gerasimou\inst{1} and Dimitris Kolovos\inst{1}}

\institute{
Department of Computer Science, University of York, UK.\\
\email{[richard.paige,simos.gerasimou,dimitris.kolovos]\_at\_york.ac.uk}
}
\maketitle

\begin{abstract}
This report describes concerns related to supporting a technical solution to software obsolescence management long-term,
specifically in the context of DSTL. It assumes the desire to support the technical solution in a context where safety-critical
software systems are being developed, maintained and certified. It identifies three categories of concerns: those related
to dependencies on the ecosystem in which the solution has been implemented; those related to programming language
evolution; and those related to qualification (which are discussed in more detail in a companion report).
\end{abstract}

\section{Introduction}

% Purpose of report - technical tender says "approach to supporting the solution long term", so we give concerns
% Structure of report

This report describes concerns related to supporting a technical solution to software obsolescence management long-term.
The technical solution is based on transformations -- extraction transformations from a legacy application to an abstract
specification (an abstract syntax tree), and generation transformations that re-host the application to a new software or
hardware platform. The solution is described in more detail in a companion report \cite{DSTL-Report}.

The solution builds on existing libraries, toolkits and APIs, and is  non-trivial software system in and of
itself. Hence, there are issues pertaining to its ongoing support and maintenance. The report classifies these issues
into three categories: concerns related to dependencies of the solution on its ecosystem and supporting infrastructure
(particularly Eclipse); concerns related to programming language evolution; and concerns related to qualification. These
are discussed in more detail in the following sections.

The concerns are explicitly raised assuming that the technical solution is to be maintained in the context of DSTL, where it
may be put to use in rehosting safety-critical or safety-related software for airborne systems. While some of the concerns
related to dependencies on the ecosystem are generic (and would apply in any context), the importance and impact associated
with mitigating these concerns are considered significant for DSTL. 



\section{Concerns Related to Dependencies on Ecosystem}

% Discuss dependencies on Eclipse (CDT, Epsilon etc) that they should be aware of. Perhaps mention the
% typical Eclipse maintenance cycle and when changes may arise that may impact on maintaining the tool chain.

\section{Concerns Related to Language Evolution}

% Discuss potential evolution of front-end languages (which should not be significant as DSTL will primarily deal
% older versions of C, C++, Ada.)
%
% Discuss potential evolution of back-end/target language (i.e., new versions of C++ with new language features,
% what impact will this have on the overall tool chain).
%
% Briefly discuss potential for change in the languages used to implement the toolchain (e.g., new versions of Java)

\section{Concerns Related to Tool Qualification}

% Link to the assurance report, which should mention the risks associated with qualifying the toolchain against
% DO-331. This section should be very brief and should highlight that if any of the components of the toolchain 
% change (e.g., CDT) then qualification may be broken - but there are standard practices to address this. The main
% expense will come from qualifying the toolchain initially.%

\bibliographystyle{abbrv}
\bibliography{bibliography}
\renewcommand{\baselinestretch}{1.0}


\end{document}